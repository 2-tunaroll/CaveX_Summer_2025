\section{Problem Definition}
\label{sec:problem-definition}
The beginning of the project's problem definition phase explores the project stakeholders and the effect they have on the project. Next, the context of the CaveX robot system is explored to identify the key interfaces required to implement the desired functionality. Finally a scenario-based needs analysis is conducted to understand a typical operation of the robot and the functionality that is required by the user. A detailed list of user needs are developed from this analysis which are then transformed into system level requirements which aim to guide the technical design to meet the user's expectations.

\subsection{Stakeholder Analysis}
To gain an improved understanding of the key stakeholders, a stakeholder analysis has been conducted and is shown in Appendix \ref{app:stakeholder}. The stakeholder analysis includes a map displaying significant interactions between groups and provides a grid depicting the varying influence and interest of the stakeholders. The stakeholders that have high interest and influence are the project supervisors, including The University of Adelaide paleontologists Craig Williams and Dr Elizabeth Reed, and the university technicians who assist with implementing hardware. These stakeholders are to be worked with closely throughout the project. Other important stakeholders include the cave inhabitants, legislative groups and Naracoorte Caves' staff who have high influence on the design due to the constraints they place on the system's operation.

\subsection{System Context}
A system context diagram, shown in Appendix \ref{fig:system_context}, defines all of the external entities the robot must interact with during its operation. These interactions define key interface requirements and behaviour that guides the system's design. The system context diagram also defines the systems that place constraints on the design.

The key interface for the system design is the communication between the remote user and the CaveX robot which is a major distinction between the current and previous project iterations. There must exist a reliable communication system that allows data to be sent to a platform which can display real time cave maps which is outlined in OB5 in Table \ref{tab:objectives}. The communication system must be two-way to also allow instructions to be received from a remote operator. Some important constraints which arise from the system context diagram are the Naracoorte Caves environment and natural wildlife as they impose special design considerations, namely noise reduction and obstacle avoidance (OB4).

\subsection{User Needs}
In keeping consistent with the systems engineering approach, a set of user needs are formulated which form the basis of the engineering design considerations for this year's project. Much of the user needs developed by the 2021 and 2022 teams will be reused in this iteration and a full list of the previous iteration's stakeholder needs are shown in Appendix \ref{app:stakeholderNeeds}. A key distinction between the stakeholder requirements from previous years and the user needs from this year is that previous iterations developed requirements of all entities which interact with the system. Since these requirements have been validated by previous iterations, the focus this year is to develop needs from the direct users of the system. However, the stakeholder requirements will still form a integral part to the design process to ensure that new functionality does not adversely effect these requirements.

To understand the needs of the end user, a typical use case of the robot system was considered through the scenario-based needs analysis shown in Appendix \ref{app:scenario-based-needs-analysis}. From this analysis and the previous stakeholder requirements, the 2023 user needs were collated and summarised in Appendix \ref{app:2023userneeds}. The needs are traceable to invalidated requirements from previous project iterations, but they have a refined focus on autonomy to align them with the CaveX mission statement. Additionally, some of the invalidated user requirements have not been included in this iteration since they are out of the scope of the project and do not directly relate to its technical objectives. These primarily include hardware-oriented upgrades, although hardware upgrades are within the scope of the project, they will not be considered unless software necessitates their inclusion.

\subsection{System Requirements}
The user needs serve as the fundamental basis of the system requirements. A complete list of the system requirements from 2022, which focused on detailed hardware design, are included in Appendix \ref{app:systemRequirements}. The additional system requirements that arise from this year's project scope are shown in Appendix \ref{app:2023sysreqs}. These requirements were derived partly from the 2022 iteration's unverified requirements. Some requirements were not verified by the 2022 team but are not included in the 2023 requirements as they do not fall under this year's project scope, do not significantly effect the quality of the project, and do not stem directly from an essential user need. This year's system requirements are integral in the project meeting the technical objectives displayed in Table \ref{tab:objectives}.

% Section: Existing Prototype
\import{sections/}{existingPrototype.tex}

% Section: Prototype Performance Analysis
\import{sections/}{prototypePerformanceAnalysis.tex}