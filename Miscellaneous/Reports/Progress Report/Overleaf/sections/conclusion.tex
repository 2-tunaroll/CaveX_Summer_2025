\section{Conclusion}
This report presented the current state of the 2023 CaveX project iteration. The technical objectives were outlined in Section \ref{aims_scope_objectives} and form the fundamental basis for this iteration's project scope. To ensure that the technical objectives are met, a systems engineering methodology has been used. This approach facilitates the ability for the final design to fulfill the user's vision by constructing user needs and system requirements. The usage of an agile engineering approach in combination with the systems engineering methodology has also been outlined to improve the flexibility in project schedule.

The key theme of the 2023 CaveX project is to implement the current robot prototype with autonomous capabilities. The literature review in Section \ref{sec:litreview} discusses the current state of autonomous algorithms for robotic control, which is a key milestone for achieving autonomous functionality. Table \ref{tab:lit_review_outcomes} shows a summarised comparison between the findings from multiple sources in regards to themes which relate to the technical objectives. From this comparison, outcomes have been highlighted which direct the implementation specifics to meet the technical objectives.

A complete problem definition has been presented in Section \ref{sec:problem-definition}. The information in this section highlights the systems engineering methodology and applies a stakeholder analysis to formulate a set of user needs. These user needs were translated into a set of system requirements which fulfill these needs. An analysis of the existing prototype is also presented. Initially, a theoretical analysis of the prototype exposed technical limitations. Subsequently, preliminary incline and decline testing was undertaken in the EXTERRES laboratory at The University of Adelaide. From this testing, the strengths and limitations of the gaits were highlighted. An analysis on the key limitations of the current prototype and identification of key prototype issues is displayed in Section \ref{sec:prototype-issues}. These limitations are consistent with the technical objectives of the project. A preliminary design has been outlined in Section \ref{sec:prelim-design} which provides the rationale for technical decisions to mitigate these limitations. Lastly, a completion plan has been outlined which illustrates the future work required to meet the objectives' deadlines shown in Table \ref{tab:objectives} and Appendix \ref{app:ganttChart}. The current progress on the technical objectives is visually displayed in Figure \ref{fig:progress-bars}. The agile development approach, elaborated on in Section \ref{agile_software_engineering}, has allowed for progress to be made on each of the objectives in parallel rather than serially. This approach will allow the objectives to be met to the desired standards highlighted in Appendix \ref{app:qualityplan} by their deadlines.